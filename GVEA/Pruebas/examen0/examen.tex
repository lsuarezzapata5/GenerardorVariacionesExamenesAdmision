\documentclass[a4paper,10pt]{article}
\usepackage[utf8]{inputenc} 
\usepackage[left=1.5cm,top=1cm,right=1.5cm,bottom=1cm]{geometry} 
\usepackage[spanish]{babel}
\usepackage{times}
\usepackage{graphicx}


\title{ 
\begin{minipage}{12cm} 
\centerline {\includegraphics{../../escudo.jpg}} 
\begin{center}Vicerector\'ia de Docencia\end{center}Departamento de Admisiones y Registro\newline\newline\newline\newline\centerline {Examen de Admisi\'on}\newline\newline\centerline {2009 - 1}\date{Jornada :  Mañana}\author{Universidad de Antioquia}\end{minipage}}\begin{document}

\maketitle
\twocolumn 
\newpage 
\noindent 
\textbf{ COMPETENCIA LECTORA } \newline \textbf{ Texto Uno} \newline
1. Gracias a la literatura, a las conciencias que formó, a los deseos y anhelos que inspiró, al desencanto de lo real con que volvemos del viaje a una bella fantasía, la civilización es ahora menos cruel que cuando los contadores de cuentos comenzaron a humanizar la vida con sus fábulas. Seríamos peores de lo que somos sin los buenos libros que leímos, más conformistas, menos inquietos e insumisos y el espíritu crítico, motor del progreso, ni siquiera existiría. Igual que escribir, leer es protestar contra las insuficiencias de la vida. Quien busca en la ficción lo que no tiene, dice, sin necesidad de decirlo, ni siquiera saberlo, que la vida tal como es no nos basta para colmar nuestra sed de absoluto, fundamento de la condición humana, y que debería ser mejor. Inventamos las ficciones para poder vivir de alguna manera las muchas vidas que quisiéramos tener cuando apenas disponemos de una sola. \newline 2. Sin las ficciones seríamos menos conscientes de la importancia de la libertad para que la vida sea vivible y del infierno en que se convierte cuando es \underline{ conculcada}  por un tirano, una ideología o una religión. Quienes dudan de que la literatura, además de sumirnos en el sueño de la belleza y la felicidad, nos alerta contra toda forma de opresión, pregúntense por qué todos los regímenes empeñados en controlar la conducta de los ciudadanos de la cuna a la tumba, la temen tanto que establecen sistemas de censura para reprimirla y vigilan con tanta suspicacia a los escritores independientes. Lo hacen porque saben el riesgo que corren dejando que la imaginación discurra por los libros; lo \underline{ sediciosas}  que se vuelven las ficciones cuando el lector coteja la libertad que las hace posibles, y que en ellas se ejerce, con el oscurantismo y el miedo que lo acechan en el mundo real. Lo quieran o no, lo sepan o no, los fabuladores, al inventar historias, propagan la insatisfacción, mostrando que el mundo está mal hecho, que la vida de la fantasía es más rica que la de la rutina cotidiana. Esa comprobación, si echa raíces en la sensibilidad y la conciencia, vuelve a los ciudadanos más difíciles de manipular, de aceptar las mentiras de quienes quisieran hacerles creer que, entre barrotes, inquisidores y carceleros viven más seguros y mejor. \newline 3. La buena literatura tiende puentes entre gentes distintas y, haciéndonos gozar, sufrir o sorprendernos, nos une por debajo de las lenguas, creencias, usos, costumbres y prejuicios que nos separan. Cuando la gran ballena blanca sepulta al capitán Ahab en el mar, se encoge el corazón de los lectores idénticamente en Tokio, Lima o Tombuctú. Cuando Emma Bovary se traga el arsénico, Anna Karenina se arroja al tren y Julián Sorel sube al patíbulo, y cuando, en El Sur, el urbano doctor Juan Dahlmann sale de aquella pulpería de la pampa a enfrentarse al cuchillo de un matón, o advertimos que todos los pobladores de Comala, el pueblo de Pedro Páramo, están muertos, el estremecimiento es semejante en el lector que adora a Buda, Confucio, Cristo, Alá o es un agnóstico, vista saco y corbata, chilaba, kimono o bombachas. La literatura crea una fraternidad dentro de la diversidad humana y eclipsa las fronteras que erigen entre hombres y mujeres la ignorancia, las ideologías, las religiones, los idiomas y la estupidez […].\newline  4. La literatura es una representación falaz de la vida que, sin embargo, nos ayuda a entenderla mejor, a orientarnos por el laberinto en el que nacimos, transcurrimos y morimos. Ella nos desagravia de los reveses y frustraciones que nos inflige la vida verdadera y gracias a ella desciframos, al menos parcialmente, el jeroglífico que suele ser la existencia para la gran mayoría de los seres humanos, principalmente aquellos que alentamos más dudas que certezas, y confesamos nuestra perplejidad ante temas como la trascendencia, el destino individual y colectivo, el alma, el sentido o el sinsentido de la historia, el más acá y el más allá del conocimiento racional. \newline 5. Siempre me ha fascinado imaginar aquella incierta circunstancia en que nuestros antepasados, apenas diferentes todavía del animal, recién nacido el lenguaje que les permitía comunicarse, empezaron, en las cavernas, en torno a las hogueras, en noches hirvientes de amenazas -rayos, truenos, gruñidos de las fieras- a inventar historias y a contárselas. Aquel fue el momento crucial de nuestro destino, porque, en esas rondas de seres primitivos suspensos por la voz y la fantasía del contador, comenzó la civilización, el largo transcurrir que poco a poco nos humanizaría y nos llevaría a inventar al individuo soberano y a desgajarlo de la tribu, la ciencia, las artes, el derecho, la libertad, a escrutar las entrañas de la naturaleza, del cuerpo humano, del espacio y a viajar a las estrellas. Aquellos cuentos, fábulas, mitos, leyendas, que resonaron por primera vez como una música nueva ante auditorios intimidados por los misterios y peligros de un mundo donde todo era desconocido y peligroso, debieron ser un baño refrescante, un remanso para esos espíritus siempre en el "quién vive", para los que existir quería decir apenas comer, guarecerse de los elementos, matar y fornicar. Desde que empezaron a soñar en colectividad, a compartir los sueños, incitados por los contadores de cuentos, dejaron de estar atados a la noria de la supervivencia, un remolino de quehaceres embrutecedores, y su vida se volvió sueño, goce, fantasía y un designio revolucionario: romper aquel confinamiento y cambiar y mejorar, una lucha para aplacar aquellos deseos y ambiciones que en ellos azuzaban las vidas figuradas, y la curiosidad por despejar las incógnitas de que estaba constelado su entorno. \newline 6. Ese \underline{ proceso} nunca interrumpido se enriqueció cuando nació la escritura y las historias, además de escucharse, pudieron leerse y alcanzaron la permanencia que les confiere la literatura. Por eso, hay que repetirlo sin tregua hasta convencer de ello a las nuevas generaciones: la ficción es más que un entretenimiento, más que un ejercicio intelectual que aguza la sensibilidad y despierta el espíritu crítico. Es una necesidad imprescindible para que la civilización siga existiendo, renovándose y conservando en nosotros lo mejor de lo humano. Para que no retrocedamos a la barbarie de la incomunicación y la vida no se reduzca al pragmatismo de los especialistas que ven las cosas en profundidad pero ignoran lo que las rodea, precede y continúa. Para que no pasemos de servirnos de las máquinas que inventamos a ser sus sirvientes y esclavos. Y porque un mundo sin literatura sería un mundo sin deseos ni ideales ni desacatos, un mundo de autómatas privados de lo que hace que el ser humano sea de veras humano: la capacidad de salir de sí mismo y mudarse en otro, en otros, modelados con la arcilla de nuestros sueños. \newline 7. De la caverna al rascacielos, del garrote a las armas de destrucción masiva, de la vida tautológica de la tribu a la era de la globalización, las ficciones de la literatura han multiplicado las experiencias humanas, impidiendo que hombres y mujeres sucumbamos al letargo, al ensimismamiento, a la resignación. Nada ha sembrado tanto la inquietud, removido tanto la imaginación y los deseos, como esa vida de mentiras que añadimos a la que tenemos gracias a la literatura para protagonizar las grandes aventuras, las grandes pasiones, que la vida verdadera nunca nos dará. Las mentiras de la literatura se vuelven verdades a través de nosotros, los lectores transformados, contaminados de anhelos y, por culpa de la ficción, en permanente entredicho con la mediocre realidad. Hechicería que, al ilusionarnos con tener lo que no tenemos, ser lo que no somos, acceder a esa imposible existencia donde, como dioses paganos, nos sentimos terrenales y eternos a la vez, la literatura introduce en nuestros espíritus la inconformidad y la rebeldía, que están detrás de todas las hazañas que han contribuido a disminuir la violencia en las relaciones humanas. A disminuir la violencia, no a acabar con ella. Porque la nuestra será siempre, por fortuna, una historia inconclusa. Por eso tenemos que seguir soñando, leyendo y escribiendo, la más eficaz manera que hayamos encontrado de aliviar nuestra condición perecedera, de derrotar a la carcoma del tiempo y de convertir en posible lo imposible. \newline \textsl{ Fragmentos de VARGAS LLOSA, Mario. “Elogio de la lectura y la ficción. Discurso Nóbel, 7 diciembre de 2010”. Generación  El Colombiano. Medellín, 19 de diciembre de 2010, p. 5, 10 y 11. } \newline 
  1. La palabra \underline{ insumisos} es utilizada por el autor para referirse a una de las influencias que él considera tiene la literatura en los seres humanos; es decir, ésta los hace:\newline \newline  
 A. Muy conformistas\newline 
 B. Más rebeldes\newline 
 C. Más seguros\newline 
 D. Menos intolerantes\newline 
 \newline 
2. En la expresión “Al igual que escribir, leer es protestar contra las insuficiencias de la vida” (párrafo 1), se:\newline \newline  
 A. Hace un elogio a la lectura\newline 
 B. Reconoce la importancia de la escritura\newline 
 C. Elogia más a la lectura que a la escritura\newline 
 D. Destaca por igual a la lectura y a la escritura\newline 
 \newline 
3. ddffdgdfgfg\newline \newline  
 A. sdsdfsdfdsf\newline 
 B. dfgfdgdfg\newline 
 C. dfgdfgdfgfdg\newline 
 D. dfgdfgdfgf\newline 
 \newline 
\end{document}
